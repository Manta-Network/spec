\lsection{Requirements}{requirements}

\subsection*{Goals}

\begin{itemize}
    \item Anonymous: Registration only requires a Twitter handle and an email address, none of which needs to be linked to a participant's identity. Participants are encouraged to publicly announce their contribution, but this is not mandatory.
    \item Low barrier to entry: participation does not require sophisticated understanding of cryptography. Participants can contribute using a free, open source, independently audited client.
    \item Low downtime: a ceremony coordinator will minimize the waiting time between contributions.
    \item Updatable: More contributions can be added to the ceremony at a later stage if desired.
    \item Verifiable: The ceremony and all its contributions can be verified by independent auditors. We ensure that is the case by:
    \begin{itemize}
        \item Publishing the transcript of the ceremony, including the validity proof and hash of each contribution.
        \item Asking participants to publish their hashes in independent locations, e.g. Twitter.
        \item Distributing an open-source verification library. 
    \end{itemize}
\end{itemize}

\subsection*{Non-Goals}

\begin{itemize}
    \item Permisionless: We require participants to pre-register to contribute to the ceremony. However, registration is open to anyone with a Twitter account and an email address.
    \item Support for independently computed contributions: We make no special effort to support contribution clients other than our own. However, the server code is open-source and a knowledgeable user could write their own contribution client, provided it adheres to the messaging protocol \ref{sec: MessagingProtocol}.
\end{itemize}