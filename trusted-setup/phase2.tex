\lsection{Groth 16 Phase 2 MPC}{phase2}\label{sec: mpc_def}

Here we define a secure, updateable, verifiable multi-party computation for the Groth16 \Setup{} function. We closely follow the protocol defined in \cite{bowe19}.

The MPC proceeds in $N$ Rounds of computation, each of which produces a \emph{state} $\sigma_n$, \emph{challenge} $c_n$, and \emph{proof} $\pi_n$. The triple $(\sigma_n, c_n, \pi_n)$ is computed recursively from the previous round's triple $(\sigma_{n-1}, c_{n-1}, \pi_{n-1})$ via functions \contribute{} and \challenge{}. A third function \verify{} allows any third party to verify from a transcript $T = \{ \sigma_n, c_n, \pi_n \}_{n=1,\ldots, N}$ that each participant computed \contribute{} correctly. 

\begin{definition}[State]
    The \emph{state} of the MPC is a Groth16 prover key \eqref{eq: prover_key}. We may denote this as 
    \begin{equation}\label{eq: short sigma} \sigma = \left( \subsigma{}, \ldots \right), \end{equation} where $\textrm{cross-term}_i = \beta u_i(x) + \alpha v_i(x) + w_i(x)$. 
    The expression above is just a shorthand for \eqref{eq: prover_key} that reminds us which parts of the prover key depend on $\delta$. Only these parts of the prover key are affected by the MPC; all terms summarized by the ellipsis remain invariant.
\end{definition}

\begin{definition}[Challenge Hash]
    \emph{Challenge hashes} are outputs of a collision-resistant hash (CRH) function $H$\footnote{We use Blake2 with a 64-byte digest, see \cite{blake2}}, see \challenge{} below.
\end{definition}

% ? Would `Discrete Log Knowledge Proof' be a more descriptive name? 
\begin{definition}[Consistent Pair]
    A \emph{consistent pair} in the bilinear group $\Gone \times \Gtwo \to \Gt$ of size $r$ consists of two pairs $(a_1, b_1) \in \Gone \times \Gone $, $(a_2, b_2) \in \Gtwo \times \Gtwo$ satisfying 
    \begin{equation}\label{eq: ratio}
        e\left( a_1, b_2 \right) = e\left( a_2, b_1\right)
    \end{equation}
\end{definition}
Consistent pairs can be used to prove the knowledge of discrete logs:

The main idea is that the pairs satisfy \eqref{eq: ratio} only if there exists a scalar $\delta$ such that $b_1 = \delta \cdot a_1$ and $b_2 = \delta \cdot a_2$ (due to the non-degeneracy of $e$). Thus the following interactive protocol requires the \Prover{} to know the discrete log $\delta$ in order to pass:
\begin{enumerate}
    \item \Prover{} sends $(a_1, b_1)$ to \Verifier{}.
    \item \Verifier{} sends challenge point $a_2 \in \Gtwo$ to \Prover{}.
    \item \Prover{} sends matching point $b_2 \in \Gtwo$ to \Verifier{}.
    \item \Verifier{} checks \eqref{eq: ratio}.
\end{enumerate}
In Step 3, a prover with knowledge of $\delta$ merely computes $b_2 = \delta \cdot a_2$. A prover without knowledge of $\delta$ must first solve a discrete log problem to find $b_2$, so this proof is sound for $r$ large enough.

In practice, step 2 above is replaced by a Fiat-Shamir transform. % todo \ref{def: challenge}
When this is the case, we define
\begin{definition}[Ratio Proof]
    A \emph{ratio proof} is a pair $(a_1, b_1) \in \Gone \times \Gone$ and a \emph{matching point} $b_2 \in \Gtwo$, together with a prescription for computing a \emph{challenge point} $a_2 \in \Gtwo$ from $a_1, b_1$ and a challenge hash, such that \eqref{eq: ratio} is satisfied. We will denote ratio proofs by $\pi$.
\end{definition}

In addition to making the above protocol non-interactive, the Fiat-Shamir step forms a \emph{contribution chain} by computing the challenge for Round $n$ from the data of Round $n-1$. Verifying the MPC means checking that state $\sigma_n$ was computed from state $\sigma_{n-1}$ using some $\delta$ attested to by a ratio proof $\pi_n$ whose $a_2$ challenge point is computed from (a commitment to) $\sigma_{n-1}, \sigma_n, \pi_{n-1}$, and the Round $n-1$ challenge point. % see verify def. below

We now define the functions needed for the MPC.

\subsubsection*{fn \initialize{}}

Inputs: Phase 1 output \eqref{eq: kzg} and R1CS circuit description.

Outputs: Groth 16 proving key $\sigma_0$ \eqref{eq: prover_key} with $\delta = 1$, challenge hash $c_0 = [0, 0, \ldots 0]$.

Definition:
\begin{enumerate}
    \item Compute commitments to the Lagrange basis polynomials for the set of $2^k$-th roots of unity (where $2^k \ge n$ is the least power of two with this property). 
    
    Specifically, these are commitments of the form $[L_q(x)]_1$, $[L_q(x)]_2$ that we compute from the KZG parameters $[x^i]_j$ via a Fourier transform. Similarly, we compute the commitments of the form $[\alpha \cdot L_q(x)]_1$, $[\beta \cdot L_q(x)]_1$, $[\beta \cdot L_q(x)]_2$ from the KZG parameters $[\alpha \cdot x^i]_1, [\beta \cdot x^i]_j$.
    \item For each circuit input $i$ (public or private), compute a commitment to the Lagrange interpolation polynomial $w_i(X)$ defined by \eqref{eq: interpolate} from the commitments to the Lagrange polynomials. Similarly, compute these commitments for $\beta \cdot u_i$, $\alpha \cdot v_i$.
    \item Compute commitments $[x^i t(x)]_1$ from the KZG parameters.
    \item The commitments $[\alpha]_1, [\beta]_1, [\beta]_2$ are already part of the KZG parameters.
    \item Assemble the above pieces to form the commitments found in \eqref{eq: prover_key} with $\delta = 1$.
\end{enumerate}

\subsubsection*{fn \contribute{}}
Inputs: $\sigma_{n-1}, c_{n-1}$, the state and challenge hash from Round $n-1$.

Outputs: $\sigma_n, \pi_n$, the state and proof of Round $n$.

Definition: 
    \begin{enumerate}
        \item Sample $\delta_n \in \mathbb{F}_r$
        \item From $\sigma_{n-1}$ of the form \eqref{eq: short sigma}, compute 
            \begin{equation*}
                \sigma_n = \left( \newsigma , \ldots \right)
            \end{equation*}
        \item Form ratio proof $\pi_n$: Sample $a_1 \in \Gone$, compute $b_1 = \delta_n \cdot a_1$. Compute challenge point $a_2$ from $c_{n-1}, a_1, b_1$. Compute $b_2 = \delta_n \cdot a_2$ and let $\pi_n = (a_1, b_1, b_2)$.
        \item Clear $\delta_n$ from memory. (This step is not related to the correct computation of $\sigma_n, \pi_n$, but rather to the security of the MPC.)
    \end{enumerate}
The first and last steps of this protocol have famously led to participants sampling $\delta$ using randomness from Chernobyl and later smashing their computers to clear it from memory. An ambitious participant can easily modify our implementation to accommodate exotic sources of randomness.

\subsubsection*{fn \challenge{}}
Inputs: $c_{n-1}, \sigma_{n-1}, \sigma_n, \pi_n$, the state and challenge hash from Round $n-1$ and the state and ratio proof from Round $n$.

Outputs: $c_n$, the challenge hash of Round $n$.

Definition: for some CRH $H$, let $c_n = H(c_{n-1} \vert \sigma_{n-1} \vert \sigma_n \vert \pi_n)$, the hash of the concatenated byte representations of each piece of data.

\subsubsection*{fn \verify{}}
Inputs: $c_{n-1}, \sigma_{n-1}, \sigma_n, \pi_n$ as above.

Outputs: boolean, indicating whether $\sigma_n$ was formed from $\sigma_{n-1}$ according to protocol.

Definition:
\begin{enumerate}
    \item Check that the Phase 2 invariants of $\sigma_{n-1}$ and $\sigma_n$ match.
    \item Compute challenge point $a_2$ from $c_{n-1}$ and $a_1, b_1$ of $\pi_n$.
    \item Check that $a_2,$ and $\pi_n = (a_1, b_1, b_2)$ form a consistent pair \eqref{eq: ratio}.
    \item Check that $a_1, b_1$ of $\pi_n$ form a consistent pair with the $[\delta]_2$ points of $\sigma_{n-1}$ and $\sigma_n$. 
    \item Check that the $[\delta]_1$ points of $\sigma_{n-1}, \sigma_n$ form a consistent pair with their $[\delta]_2$ points.
    \item\label{testlabel} Check that the terms $\subsubsigma{}$ of $\sigma_{n-1}, \sigma_n$ form a consistent pair with the $[\delta]_2$ points. Here we reverse the order of the pair of $[\delta]_2$ points to account for the fact that $\subsubsigma{}$ is proportional to $\delta^{-1}$.
\end{enumerate}
In practice, Step \ref{testlabel} is performed quickly by computing random linear combinations of the points $\subsubsigma{}$ and checking this pair's consistency (see \cite{bowe19}).
