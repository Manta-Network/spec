\lsection{Context}{Context}

\subsection{Circuit}
Throughout this paper, by circuit we mean a \emph{Rank-1 Constraint System (R1CS)}. It is defined as a system of equations over $\mathbb{F}_r$ of the form 
\begin{align}
\sum_{i=0}^m a_i u_{i, q} \cdot \sum_{i=0}^m a_i v_{i, q} = \sum_{i=0}^m a_i w_{i, q}, \qquad q = 1, \dots, n, \label{eqn: r1cs}
\end{align}
where $a_0 = 1$. This system of equations, in the context of zero-knowledge proofs, is to be understood as follows:
\begin{itemize}
\item The numbers $u_{i, q}, v_{i, q}, w_{i, q}$ are constants in $\mathbb{F}_r$ which represent the operations performed in an arithmetic circuit. Here constant means constant in the circuit, e.g. each \MantaPay{} circuit will have a fixed set of $u_{i, q}, v_{i, q}, w_{i, q}$.
\item The numbers $\phi = (a_1, \dots, a_\ell)$ are the public inputs. In \MantaPay{}, these correspond to the \TransferPost{}, excluding the proof.
\item The numbers $w = (a_{\ell + 1}, \dots, a_m)$ are the private witnesses. In \MantaPay{}, these correspond to the elements of the \Transfer{} which are not part of the \TransferPost{}.
\item A R1CS defines the following binary relation
\begin{align}
R = \left\{ (\phi, w) \big| \ \phi = (a_1, \dots, a_\ell), \ w = (a_{\ell + 1}, \dots, a_m), \ \eqref{eqn: r1cs} \textrm{ is satisfied} \right\} \subset \mathbb{F}_r^\ell \times \mathbb{F}_r^{m-\ell}
\end{align}
\item The statements that can be proved in this terminology are of the form,  ``For a given circuit \eqref{eqn: r1cs} and public input $\phi$, I know a witness $w$ such that $(\phi, w) \in R$.''
\end{itemize}

\subsection{Quadratic Arithmetic Programs}
\emph{Quadratic Arithmetic Programs (QAPs)} give an alternative way to describe a circuit, equivalent to R1CS. A QAP is a system of polynomial equations of the form
\begin{align}
\sum_{i=0}^m a_i u_i(X) \cdot \sum_{i=0}^m a_i v_i(X) \equiv \sum_{i=0}^m a_i w_i(X) \mod t(X), \label{eqn: qap}
\end{align}
where 
\begin{itemize}
\item $u_i(X), v_i(X), w_i(X) \in \mathbb{F}_r[X]$ are degree $n-1$ polynomials, and $t(X) \in \mathbb{F}_r[X]$ is a degree $2^k$ polynomial (see below), all of which are fixed for the protocol.
\item The numbers $\phi = (a_1, \dots, a_\ell)$ are the public inputs.
\item The numbers $w = (a_{\ell + 1}, \dots, a_m)$ are the private witnesses.
\item A QAP defines the following binary relation
\begin{align}
R = \left\{ (\phi, w) \big| \ \phi = (a_1, \dots, a_\ell), \ w = (a_{\ell + 1}, \dots, a_m), \ \eqref{eqn: qap} \textrm{ is satisfied} \right\} \subset \mathbb{F}_r^\ell \times \mathbb{F}_r^{m-\ell}
\end{align}
\item The statements that can be proved in this terminology are of the form, ``For a given circuit \eqref{eqn: qap} and public input $\phi$, I know a witness $w$ such that $(\phi, w) \in R$.''
\end{itemize}

We derive the QAP description of a circuit from its R1CS description as follows:
\begin{enumerate}
    \item Choose $k$ as the minimal integer such that $2^k \geq n$. Let $t(X) = X^{2^k} - 1$. This is the \emph{vanishing polynomial} for the set of $2^k$-th roots of unity in $\mathbb{F}_r$.
    \item We derive the polynomial $u_i(X)$ from the R1CS vector $(u_{i,q})_{q=1}^n$ via a Lagrange basis $\{ L_q(x) \}_{t(q) = 0}$ for the set of $2^k$-th roots of unity.  That is,
    \begin{equation}\label{eq: interpolate}
        u_i(X) = \sum_{q=1}^n u_{i,q} L_q(X),
    \end{equation}
    where we use the convention $u_{i,q} = 0$ for $q > n$.
    \item We repeat the same procedure to define $v_i(X)$ and $w_i(X)$.
\end{enumerate}
One may readily check that \eqref{eqn: r1cs} is equivalent to \eqref{eqn: qap} with these definitions.

\subsection{The Groth16 \Setup{} function}
We now recall the Groth16 \Setup{} function. It is defined relative to two choices:
\begin{itemize}
\item A pairing curve\footnote{We use BN254 (see \cite{Naehrig10} and the references therein)}, which consists of a triple of elliptic curves $(\mathbb{G}_1, \mathbb{G}_2,\mathbb{G}_T) $ and a non-degenerate bilinear map $e: \mathbb{G}_1 \times \mathbb{G}_2 \to \mathbb{G}_T$. We fix generators $g$ and $h$ of $\mathbb{G}_1$ and $\mathbb{G}_2$, respectively. We require that $g, h$, and $e(g,h)$ all have the same prime order $r$. 
\item A circuit, encoded as a QAP $\{u_i(X), v_i(X), w_i(X), t(X)\}$.
\end{itemize}
All public parameters derive from a simulation trapdoor $\tau = (\alpha, \beta, \delta, x) \leftarrow \mathbb{F}_r^*$ (the famous ``toxic waste''). The Groth16 public parameters themselves are elements of $\Gone$ and $\Gtwo$, specifically
\begin{equation}\label{eq: prover_key}
    \begin{split} % TODO: Why are there raw powers of x in sigma2 ?
\sigma_1 = \Biggl[ \alpha, \beta, \delta, 
    \left\{u_i(x) \right\}_{i=0}^m, &
    \left\{v_i(x) \right\}_{i=0}^m, 
    \left\{ (\beta u_i(x) + \alpha v_i(x) + w_i(x)) \right\}_{i=0}^\ell, \\
    & \left\{  \frac{\beta u_i(x) + \alpha v_i(x) + w_i(x)}{\delta} \right\}_{i=\ell+1}^m, 
    \left\{ \frac{x^i t(x)}{\delta} \right\}_{i=0}^{n-2}  \Biggr]_1 \\
\sigma_2 = \left[ \left( \beta, \delta, \left\{ v_i(x) \right\}_{i=0}^m \right) \right]_2
    \end{split}
\end{equation} 
where we denote $[y]_1 = y \cdot g$ and $[y]_2 = y \cdot h$ for all $y \in \mathbb{F}_r$. These public parameters make up the Groth 16 proving and verifying keys.

The public output of \Setup{} is $\sigma = (\sigma_1, \sigma_2)$. These are the public parameters from which Groth16 proofs are formed. The trapdoor $\tau$ is \emph{not} public; indeed, a malicious prover with knowledge of $\tau$ could construct fraudulent proofs. The goal of the trusted setup is to compute $\sigma$ without revealing $\tau$.

\subsection{Multi-Party Computation}
A decentralized way to compute $\sigma$ without revealing $\tau$ is to compute $\sigma$ in such a way that $\tau$ becomes a shared secret split among a diverse set of participants. This may be achieved via \emph{secure multi-party computation} (MPC). The MPC we employ is a protocol for computing $\sigma$ incrementally from private inputs $\tau_i$ belonging to participants in the computation. 

The key security property of this MPC is that its security is ensured by having at least one honest participant. An honest participant is one who keeps their private input $\tau_i$ from all other participants, ideally by permanently clearing it from their system's memory after participation. Put differently, this \emph{1-out-of-N} honest participants guarantee states that to determine the toxic waste $\tau$ requires the collusion of \emph{all} participants in the MPC. 

By soliciting contributions to the MPC from a diverse set of participants, we increase the difficulty of such collusion. Note that any individual with a stake in the security of \MantaPay{} can guarantee this personally, simply by participating honestly in the \Setup{} MPC.


\subsection{Phase Structure}

The full \Setup{} MPC splits usefully into two \emph{phases}. The output of \emph{Phase 1} is \emph{universal} in the sense that these parameters may be used by any ZK circuit of small enough size. In \emph{Phase 2} we derive $\sigma$ from the output of Phase 1. The parameters generated in Phase 2 are \emph{circuit-specific}: they depend on the R1CS description of the circuit \eqref{eqn: r1cs} and must be computed separately for each ZK circuit. This two-phase splitting of the MPC is formalized in \cite{bowe19}.

Phase 1 computes a modified KZG setup \cite{KZG} consisting of the curve points 
\begin{align}\label{eq: kzg}
KZG = \KzgParams{}
\end{align}
These Phase 1 parameters are computed via a MPC such that $x, \alpha, \beta \in \mathbb{F}_r$ are shared secrets among the participants. Since Phase 1 parameters don't depend on the specifics of the circuit, a Phase 1 ceremony may be treated as a public good for any project using circuits of appropriate size over the same pairing curve. In our case, we take the Phase 1 parameters computed by the Perpetual Powers of Tau (PPoT) ceremony \cite{PPoT}, an ongoing multi-party computation similar to our trusted setup described below, with a $1$-out-of-$N$ security assumption. To ensure its soundness, we have personally verified each contribution to PPoT.
