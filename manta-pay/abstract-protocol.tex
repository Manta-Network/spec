\lsection{Abstract Protocol}{abstract-protocol}

\subsection{Abstract Cryptographic Schemes}

In the following section, we outline the formal specifications for all of the \emph{cryptographic schemes} used in the \MantaPay{} protocol.

\begin{definitiontoc}{Commitment Scheme}
    A \emph{commitment scheme} $\COM$ is defined by the schema:
    \begin{align*}
        \Randomness             &: \Type \\
        \Input                  &: \Type \\
        \Output                 &: \Type \\
        \commit                 &: \Randomness \times \Input \to \Output
    \end{align*}
    with the following properties:
    \begin{itemize}
        \item \textbf{Binding}: It is infeasible to find an $x, y : \Input$ and $r, s: \Randomness$ such that $x \ne y$ and $\commit(r,x) = \commit(s,y)$.
        \item \textbf{Hiding}: For all $x, y : \Input$, the distributions $\set{\commit(r,x) \,|\, r \sim \Randomness}$ and \\ $\set{\commit(r,y) \,|\, r \sim \Randomness}$ are \emph{computationally indistinguishable}.
    \end{itemize}
    \textbf{Notation}: For convenience, we may refer to $\COM.\commit(r,x)$ by $\COM(r, x)$.
\end{definitiontoc}

\begin{definitiontoc}{Hash Function}
    A \emph{hash function} $\HASH$ is defined by the schema:
    \begin{align*}
        \Input  &: \Type \\
        \Output &: \Type \\
        \hash   &: \Input \to \Output
    \end{align*}
    with the following properties:
    \begin{itemize}
        \item \textbf{Collision Resistance}: It is infeasible to find $a, b : \Input$ such that $a \ne b$ and $\hash(a) = \hash(b)$.
        \item \textbf{Pre-Image Resistance}: Given $y : \Output$, it is infeasible to find an $x : \Input$ such that $\hash(x) = y$.
        \item \textbf{Second Pre-Image Resistance}: Given $a : \Input$, it is infeasible to find another $b : \Input$ such that $a \ne b$ and $\hash(a) = \hash(b)$.
    \end{itemize}

    We can also ask that a hash function be \emph{binding} or \emph{hiding} as in the above \emph{Commitment Scheme} definition if we partition the $\Input$ space into a separate \Randomness{} and \Input{} space.

    \textbf{Notation}: For convenience, we may refer to $\HASH.\hash(x)$ by $\HASH(x)$.
\end{definitiontoc}

\begin{definitiontoc}{Signature Scheme}
    A \emph{signature scheme} $\SIG$ is defined by the schema:
    \begin{align*}
        \SigningKey   &: \Type \\
        \VerifyingKey &: \Type \\
        \Randomness   &: \Type \\
        \Message      &: \Type \\
        \Signature    &: \Type \\
        \derive       &: \SigningKey \to \VerifyingKey \\
        \sign         &: \SigningKey \times \Randomness \times \Message \to \Signature \\
        \verify       &: \VerifyingKey \times \Signature \times \Message \to \Bool
    \end{align*}
    with the following properties:
    \begin{itemize}
        \item \textbf{Correctness}: For a given $\sk : \SigningKey$, $r : \Randomness$, and $m : \Message$, we have that 
            \[\verify(\derive(\sk), \sign(\sk, r, m), m) = \True\]
    \end{itemize}
\end{definitiontoc}

\begin{definitiontoc}{Authenticated Encryption Scheme}
    An \emph{authenticated encryption} scheme $\AUTH$ is defined by the schema:
    \begin{align*}
        \Key        &: \Type \\
        \Plaintext  &: \Type \\
        \Ciphertext &: \Type \\
        \Tag        &: \Type \\
        \encrypt    &: \Key \times \Plaintext \to \Tag \times \Ciphertext \\
        \decrypt    &: \Key \times \Tag \times \Ciphertext \to \Option(\Plaintext)
    \end{align*}
    with the following properties:
    \begin{itemize}
        \item \textbf{Correctness}: For a given $k : \Key$, $p : \Plaintext$, we have that $\decrypt(k, \encrypt(k, p)) = \Some(p)$.
    \end{itemize}
\end{definitiontoc}

\begin{definitiontoc}{Dynamic Cryptographic Accumulator}
    A \emph{dynamic cryptographic accumulator} $\DCA$ is defined by the schema:
    \begin{align*}
        \Item     &: \Type \\
        \Output   &: \Type \\
        \Witness  &: \Type \\
        \StateT   &: \Type \\
        \current  &: \StateT \to \Output \\
        \insertF  &: \Item \times \StateT \to \StateT \\
        \contains &: \Item \times \StateT \to \Option(\Output \times \Witness) \\
        \verify   &: \Item \times \Output \times \Witness \to \Bool
    \end{align*}
    with the following properties:
    \begin{itemize}
        \item \textbf{Unique Accumulated Values}: For any initial state $s : \StateT$ and any list of items $I : \List(\Item)$ we can generate the sequence of states:
            \[s_0 := s, \,\,\,\,\, s_{i + 1} := \insertF(I_i, s_i)\]
            Then, if we collect the accumulated values for these states, $z_i := \current(s_i)$, there should be exactly $\abs{I}$-many unique values, one for each state update.
        \item \textbf{Provable Membership}: For any initial state $s : \StateT$ and any list of items $I : \List(\Item)$ we can generate the sequences of states:
            \[s_0 := s, \,\,\,\,\, s_{i + 1} := \insertF(I_i, s_i)\]
            Then, if we collect the states $s_i$ into a set $S$, we have the following property for all $s \in S$ and $t \in I$,
            \[\Some(z, w) := \contains(t, s), \,\,\,\,\, \verify(t, z, w) = \True\]
    \end{itemize}
\end{definitiontoc}

\begin{definitiontoc}{Non-Interactive Zero-Knowledge Proving System}
    A \emph{non-interactive zero-knowledge proving system} $\NIZK$ is defined by the schema:
    \begin{align*}
        \Statement    &: \Type \\
        \ProvingKey   &: \Type \\
        \VerifyingKey &: \Type \\
        \PublicInput  &: \Type \\
        \SecretInput  &: \Type \\
        \Proof        &: \Type \\
        \keys         &: \Statement \to \mathfrak{D}(\ProvingKey \times \VerifyingKey) \\
        \prove        &: \Statement \times \ProvingKey \times \PublicInput \times \SecretInput \to \mathfrak{D}(\Option(\Proof)) \\
        \verify       &: \VerifyingKey \times \PublicInput \times \Proof \to \Bool
    \end{align*}
    \textbf{Notation}: We use the following notation for a $\NIZK$:
    \begin{itemize}
        \item We write the $\Statement$ and $\ProvingKey$ arguments of $\prove$ in the superscript and subscript respectively,
            \[\prove^P_\pk(x, w) := \prove(P, \pk, x, w)\]
        \item We write the $\VerifyingKey$ argument of $\verify$ in the subscript,
            \[\verify_\vk(x, \pi) := \verify(\vk, x, \pi)\]
        \item We say that $(x, w) : \PublicInput \times \SecretInput$ has the property of being a $\satisfying$ input whenever
            \[\satisfying^P_\pk(x, w) := \exists \pi : \Proof,\, \Some(\pi) \in \prove^P_\pk(x, w)\]
    \end{itemize}

    Every $\NIZK$ has the following properties for a fixed statement $P : \Statement$ and keys $(\pk, \vk) \sim \keys(P)$:

    \begin{itemize}
        \item \textbf{Completeness}: For all $(x, w) : \PublicInput \times \SecretInput$, if $\satisfying^P_\pk(x, w) = \True$ with proof witness $\pi$, then $\verify_\vk(x, \pi) = \True$.
        \item \textbf{Knowledge Soundness}: For any polynomial-size adversary $\mathcal{A}$,
            \[\mathcal{A} : \ProvingKey \times \VerifyingKey \to \mathfrak{D}(\PublicInput \times \Proof)\]
             there exists a polynomial-size extractor $\mathcal{E}_\mathcal{A}$
            \[\mathcal{E}_\mathcal{A} : \ProvingKey \times \VerifyingKey \to \mathfrak{D}(\SecretInput)\]
            such that the following probability is negligible:
            \[
                \Prob{
                    \satisfying^P_\pk(x, w) = \False \\
                    \verify_\vk(x, w) = \True
                }{
                    (\pk, \vk) \sim \keys(P) \\
                    (x, \pi) \sim \mathcal{A}(\pk, \vk) \\
                    w \sim \mathcal{E}_\mathcal{A}(\pk, \vk)
                }
            \]
        \item \textbf{Statistical Zero-Knowledge}: There exists a stateful simulator $\mathcal{S}$, such that for all stateful distinguishers $\mathcal{D}$, the difference between the following two probabilities is negligible:
            \[
                \Prob{
                    \satisfying^P_\pk(x, w) = \True \\
                    \mathcal{D}(\pi) = \True
                }{
                    (\pk, \vk) \sim \keys(P) \\
                    (x, w) \sim \mathcal{D}(\pk, \vk) \\
                    \Some(\pi) \sim \prove^P_\pk(x, w)
                }
                \,\,\text{and}\,\,
                \Prob{
                    \satisfying^P_\pk(x, w) = \True \\
                    \mathcal{D}(\pi) = \True
                }{
                    (\pk, \vk) \sim \mathcal{S}(P) \\
                    (x, w) \sim \mathcal{D}(\pk, \vk) \\
                    \pi \sim \mathcal{S}(x)
                }
            \]
        \item \textbf{Succinctness}: For all $(x, w) : \PublicInput \times \SecretInput$, if $\Some(\pi) \sim \prove(P, \pk, x, w)$, then $\abs{\pi} = \mathcal{O}(1)$, and $\verify(\vk, x, \pi)$ runs in time $\mathcal{O}(\abs{x})$.
    \end{itemize}
\end{definitiontoc}

\begin{definitiontoc}{Cryptographic Group}
    We define a \emph{cryptographic group} $(\G, p, g)$ as some finite cyclic group $\G$, of prime order $p$ with generator $g$ where the discrete logarithm problem is hard, namely, given $X \in \G$ it is infeasible to find $x$ such that $X = g^x$. We may omit the prime $p$ when convenient.
\end{definitiontoc}

\lsubsection{Addresses and Key Components}{addresses-and-key-components}

For the \Transfer{} protocol we use a multi-layered system of keys:

\begin{center}
    \vspace{1em}
    \begin{mdframed}[leftmargin=0.125\textwidth, rightmargin=0.125\textwidth]
        \begin{center}
            \begin{tikzcd}
                &  & \sk \arrow[lldd, "(\dash)\,\cdot\,\alpha"'] \arrow[rrdd, "g\,\cdot\,(\dash)"] & & & & & & \\
                & & & & & & & & \\
                \sk_\alpha \arrow[rrdd, "g\,\cdot\,(\dash)"'] & & & & \ak \arrow[lldd, "(\dash)\,\cdot\,\alpha"] \arrow[rr, "\KDF^\vk"] & & \vk \arrow[rr, "g\,\cdot\,(\dash)"] & & \pk \\
                & & & & & & & & \\
                & & \ak_\alpha & & & & & &
\end{tikzcd}
        \end{center}
    \end{mdframed}
    \vspace{-1em}
    \captionof{figure}{Detailed Key Schedule for \MantaPay{} where $\alpha$ is a random scalar and $g$ is a generator.}
\end{center}

Here we define each key and its function in the \Transfer{} protocol:

\begin{definition}[Key Schedule]
    A $\KeySchedule$ is a collection of implementations of the following abstract cryptographic primitives as described in the above definitions:
    \begin{itemize}
        \item \textbf{Cryptographic Group}: $(\G, p, g)$
        \item \textbf{Viewing Key Derivation Function}: $\KDF^\vk$
        \item \textbf{Proof Authorization Signature}: $\SIG$
    \end{itemize}
    with the following notational conventions:
    \begin{align*}
        \SpendingKey         &:= Z_p \\
        \ProofAuthorizingKey &:= \G \\
        \ViewingKey          &:= Z_p \\
        \zkAddress           &:= \G
    \end{align*}
    with the following constraints:
    \begin{align*}
        \SIG.\SecretKey &= \Z_p \\
        \SIG.\PublicKey &= \G \\
        \SIG.\derive    &= g\,\cdot\,(\dash)
    \end{align*}
    To derive the \zkAddress{}, $\pk$, we use the following:
    \[\sk \quad\mapsto\quad \ak := g \cdot \sk \quad\mapsto\quad \vk := \KDF^\vk(\ak) \quad\mapsto\quad \pk := g \cdot \vk\]
    For signing a message $m$ with a randomized key, the owner of the \SpendingKey{}, $\sk$, and owner of the \ProofAuthorizingKey{}, $\ak$, perform the following protocol:
    \begin{enumerate}
        \item Spender samples $\alpha$ randomly and sends it to prover.
        \item Prover computes $\ak_\alpha := \ak \cdot \alpha$ and binds it to the message $m$ and sends the message to spender.
        \item Spender computes $\sk_\alpha := \sk \cdot \alpha$ and checks that $\ak_\alpha = g \cdot \sk_\alpha$ and signs the message $m$ with $\sk_\alpha$.
    \end{enumerate}
\end{definition}

\lsubsection{\Transfer{} Protocol}{transfer-protocol}

The \Transfer{} protocol is the fundamental abstraction in \MantaPay{} and facilitiates the valid transfer of \zkAsset{s} among participants while preserving their anonymity. The \Transfer{} is made up of special cryptographic constructions called \Sender{s} and \Receiver{s} which represent the private input and the private output of a transaction. To perform a \Transfer{}, a protocol participant gathers the \SpendingKey{s} they own, selects a subset of the \UTXO{s} they have still not spent (with a fixed \AssetId{}), collects \ReceivingKey{s} from other participants for the outputs, assigning each key a subset of the input \zkAsset{s}, and then builds a \Transfer{} object representing the transfer they want to build. From this \Transfer{} object, they construct a \TransferPost{} which they then send to the \Ledger{} to be validated and stored, representing a completed state transition in the \Ledger{}. The transformation from \Transfer{} to \TransferPost{} involves keeping the parts of the \Transfer{} that \emph{must} be known to the \Ledger{} and for the parts that \emph{must} not be known, substituting them for a \emph{zero-knowledge proof} representing the validity of the secret information known to the participant, and the \Transfer{} as a whole.

We begin by defining the cryptographic primitives involved in the \Transfer{} protocol:

\begin{definition}[Transfer Configuration]
    A \TransferConfiguration{} is a collection of implementations of the following abstract cryptographic primitives:
    \begin{itemize}
        \item \textbf{Key Schedule}: $\KeySchedule$
        \item \textbf{Incoming Authenticated Encryption Scheme}: $\AUTHin$
        \item \textbf{Outgoing Authenticated Encryption Scheme}: $\AUTHout$
        \item \textbf{UTXO Commitment Scheme}: $\COM^{\UTXO} : \F \times \G \times \Asset \to \F$
        \item \textbf{Nullifier Commitment Scheme}: $\COM^{\textsf{N}} : \G \times \F \to \F$
        \item \textbf{UTXO Hash Function}: $H^{\UTXO} : \Bool \times \Asset \times \F \to \F$
        \item \textbf{UTXO Dynamic Cryptographic Accumulator}: $\UTXOSet$
        \item \textbf{Zero-Knowledge Proving System}: $\NIZK$
    \end{itemize}
    where $\F$ is a prime field and $\G$ is a cryptographic group. The \Nullifier{} type is the output of $\COM^{\textsf{N}}$ and the \UTXO{} type is the tuple $\Bool \times \Asset \times \F$ where the last field element is the output of $\COM^{\UTXO}$.
\end{definition}

For the rest of this section, we assume the existence of a \TransferConfiguration{} and use the primitives outlined above explicitly. We continue by defining the \Sender{} and \Receiver{} constructions as well as their public counterparts, the \SenderPost{} and \ReceiverPost{}.

\begin{definition}[\Transfer{} Sender]
    A \Sender{} is the following tuple:
    \begin{align*}
        r      &: \F \\
        \sa    &: \Asset \\
        \pa    &: \Asset \\
        t      &: \Bool \\
        \asset &: \Asset \\
        \cm    &: \F \\
        \utxo  &: \UTXO \\
        h      &: \F \\
        h_z    &: \UTXOSet.\Output \\
        h_w    &: \UTXOSet.\Witness \\
        n      &: \Nullifier \\
        \esk   &: \Z_p \\
        \epk   &: \G \\
        \Cout  &: \AUTHout.\Ciphertext
    \end{align*}

    A \Sender{}, $S$, is constructed from a public key $\pk : \zkAddress$ with the following algorithm:
    \begin{align*}
        t      &:= \iszero(\sa) \\
        \asset &:= \select(t, \sa, \pa) \\
        \cm    &:= \COM^{\UTXO}(r, \pk, \sa) \\
        \utxo  &:= (t, \pa, \cm) \\
        h      &:= H^{\UTXO}(\utxo) \\
        \Some\left(h_z, h_w\right) &:= \UTXOSet.\contains(h, \Ledger.\utxos()) \\
        n      &:= \COM^{\textsf{N}}(\ak, h) \\
        \epk   &:= g \cdot \esk \\
        \Cout  &:= \AUTHout.\encrypt(\pk \cdot \esk, \select(t, \sa, \pa))
    \end{align*}
\end{definition}

\begin{definition}[\Transfer{} Sender Post]
    A \SenderPost{} is the following tuple extracted from a \Sender{}:
    \begin{align*}
        h_z   &: \UTXOSet.\Output \\
        n     &: \Nullifier \\
        \epk  &: \G \\
        \Cout &: \AUTHout.\Ciphertext
    \end{align*}
    which are the parts of a \Sender{} which should be \emph{posted} to the \Ledger{}.
\end{definition}

\begin{definition}[\Transfer{} Receiver]
    A \Receiver{} is the following tuple:
    \begin{align*}
        \pk    &: \zkAddress \\
        r      &: \F \\
        \sa    &: \Asset \\
        \pa    &: \Asset \\
        t      &: \Bool \\
        \asset &: \Asset \\
        \cm    &: \F \\
        \utxo  &: \UTXO \\
        \esk   &: \Z_p \\
        \epk   &: \G \\
        \Cin   &: \AUTHin.\Ciphertext
    \end{align*}
    
    A \Receiver{}, $R$, is constructed in the following way:
    \begin{align*}
        t      &:= \iszero(\sa) \\
        \asset &:= \select(t, \sa, \pa) \\
        \cm    &:= \COM^{\UTXO}(r, \pk, \sa) \\
        \utxo  &:= (t, \pa, \cm) \\
        \epk   &:= g \cdot \esk \\
        \Cin   &:= \AUTHin.\encrypt(\pk \cdot \esk, (r, \sa))
    \end{align*}
\end{definition}

\begin{definition}[\Transfer{} Receiver Post]
    A \ReceiverPost{} is the following tuple extracted from a \Receiver{}:
    \begin{align*}
        \utxo &: \UTXO \\
        \epk  &: \G \\
        \Cin  &: \AUTHin.\Ciphertext
    \end{align*}
    which are the parts of a \Receiver{} which should be \emph{posted} to the \Ledger{}.
\end{definition}

\begin{definition}[\Transfer{} Sources and Sinks]
    A \Source{} (or a \Sink{}) is an \Asset{} representing a public input (or output) of a \Transfer{}.
\end{definition}

\begin{definition}[\Transfer{} Object]
    A \Transfer{} is the following tuple:
    \begin{align*}
        \ID        &: \Option(\AssetId) \\
        \sources   &: \List(\AssetValue) \\
        \senders   &: \List(\Sender) \\
        \receivers &: \List(\Receiver) \\
        \sinks     &: \List(\AssetValue)
    \end{align*}
    The \emph{shape} of a \Transfer{} is the following $4$-tuple of cardinalities of those sets
    \[\left(\abs{T.\sources}, \abs{T.\senders}, \abs{T.\receivers}, \abs{T.\sinks}\right)\]
    Also, note that the $\ID$ value is optional. This is inhabited whenever there are $\sources$ or $\sinks$, but if the shape of the transaction is $(0, m, n, 0)$ then $\ID = \None$.
\end{definition}

In order for a \Transfer{} to be considered \emph{valid}, it must adhere to the following constraints:

\begin{itemize}
    \item \textbf{Correct Key Signing}: The keys used to construct \Sender{s} and \Receiver{s} are valid and can be signed by a unique \SpendingKey{}.
    \item \textbf{Same Id}: All the \AssetId{s} in the \Transfer{} must be equal.
    \item \textbf{Balanced}: The sum of input \AssetValue{s} must be equal to the sum of output \AssetValue{s}.
    \item \textbf{Well-formed Senders}: All of the \Sender{s} in the \Transfer{} must be constructed according to the above \Sender{} definition.
    \item \textbf{Well-formed Receivers}: All of the \Receiver{s} in the \Transfer{} must be constructed according to the above \Receiver{} definition.
\end{itemize}

In order to prove that these constraints are satisfied for a given \Transfer{}, we build a zero-knowledge proof which will witness that the \Transfer{} is valid and should be accepted by the \Ledger{}.

\begin{definition}[\Transfer{} Validity \Statement{}]\label{def:transfer-validity-statement}
    A transfer $T : \Transfer$ is considered \emph{valid} if and only if
    \begin{enumerate}
        \item The signing authority is correctly constructed:
            \begin{align*}
                \ak_\alpha &:= \ak \cdot \alpha \\
                \vk        &:= \KDF^{\vk}(\ak) \\
                \pk        &:= g \cdot \vk
            \end{align*}
        \item All the \AssetId{s} in $T$ are equal:
            \[
                \abs{\,
                    T.\ID
                    \cup
                    \left(\bigcup_{S \in T.\senders} S.\asset.\ID \right)
                    \cup
                    \left(\bigcup_{R \in T.\receivers} R.\asset.\ID \right)
                \,} = 1
            \]
        \item The sum of input \AssetValue{s} is equal to the sum of output \AssetValue{s}:
            \[
                \left(\sum_{a \in T.\sources} a\right)
                +
                \left(\sum_{S \in T.\senders} S.\asset.\VALUE\right)
                =
                \left(\sum_{R \in T.\receivers} R.\asset.\VALUE\right)
                +
                \left(\sum_{a \in T.\sinks} a\right)
            \]
        \item For all $S \in T.\senders$, the \Sender{} $S$ is well-formed:
            \begin{align*}
                S.t     &:= \iszero(\sa) \\
                S.\asset &:= \select(S.t, S.\sa, S.\pa) \\
                S.\cm   &:= \COM^{\UTXO}(S.r, S.\pk, S.\sa) \\
                S.\utxo &:= (S.t, S.\pa, S.\cm) \\
                S.h     &:= H^{\UTXO}(S.\utxo) \\
                \UTXOSet.\verify(S.h, S.h_z, S.h_w) &= \True \\
                S.n     &:= \COM^{\textsf{N}}(\ak, S.h) \\
                S.\epk  &:= g \cdot S.\esk \\
                S.\Cout &:= \AUTHout.\encrypt(\pk \cdot S.\esk, S.\asset)
            \end{align*}
        \item For all $R \in T.\receivers$, the \Receiver{} $R$ is well-formed:
            \begin{align*}
                R.t      &:= \iszero(R.\sa) \\
                R.\asset &:= \select(R.t, R.\sa, R.\pa) \\
                R.\cm    &:= \COM^{\UTXO}(R.r, R.\pk, R.\sa) \\
                R.\utxo  &:= (R.t, R.\pa, R.\cm) \\
                R.\epk   &:= g \cdot R.\esk \\
                R.\Cin   &:= \AUTHin.\encrypt(R.\pk \cdot R.\esk, (R.r, R.\sa))
            \end{align*}
    \end{enumerate}
    \textbf{Notation}: This statement is denoted $\ValidTransfer$ and is assumed to be expressible as a \Statement{} of \NIZK{}.
\end{definition}

To finish the transfer, the \SpendingKey{} for the $\Transfer.\ak : \ProofAuthorizingKey$ needs to sign the public side of the transaction. The public part of the transaction is the following post body:

\begin{definition}[\Transfer{} Post Body]
    A \TransferPostBody{} is the following tuple:
    \begin{align*}
        \ID        &: \Option(\AssetId) \\
        \sources   &: \List(\Source) \\
        \senders   &: \List(\SenderPost) \\
        \receivers &: \List(\ReceiverPost) \\
        \sinks     &: \List(\Sink) \\
        \pi        &: \NIZK.\Proof 
    \end{align*}
    A \TransferPostBody{}, $B$, is constructed by assembling the zero-knowledge proof of \Transfer{} validity from a known proving key $\pk : \NIZK.\ProvingKey$ and a given $T : \Transfer$:
   \begin{align*}
        x            &:= \Transfer.\public(T) \\
        w            &:= \Transfer.\secret(T) \\
        \Some(\pi)   &\sim \NIZK.\prove^\ValidTransfer_\pk(x, w) \\
        B.\ID        &:= x.\ID \\
        B.\sources   &:= x.\sources \\
        B.\senders   &:= x.\senders \\
        B.\receivers &:= x.\receivers \\
        B.\sinks     &:= x.\sinks \\
        B.\pi        &:= \pi
    \end{align*}
    where $\Transfer.\public$ returns \SenderPost{s} for each \Sender{} in $T$ and \ReceiverPost{s} for each \Receiver{} in $T$, keeping \Source{s} and \Sink{s} as they are, and $\Transfer.\secret$ returns all the rest of $T$ which is not part of the output of $\Transfer.\public$.
\end{definition}

Now we can sign this body with $\sk_\alpha : \SpendingKey := \sk \cdot \alpha$ where the signature scheme has $\TransferPostBody$ as the $\SIG.\Message$ type and we use $\ak_\alpha$ as the verifying key:

\begin{definition}[\Transfer{} Post]
    A \TransferPost{} is the following tuple:
    \begin{align*}
        \sigma &: \Option(\SIG.\VerifyingKey \times \SIG.\Signature) \\
        \body  &: \TransferPostBody
    \end{align*}
    Note that the $\sigma$ value is optional. This is inhabited whenever the number of \Sender{s} in a transaction is positive.
\end{definition}

Now that a participant has constructed a transfer post $P : \TransferPost$ they can send it to the \Ledger{} for verification. 

\begin{definition}[\Ledger{}-side \Transfer{} Validity]
    To check that $P$ represents a valid \Transfer{}, the ledger checks the following:
    \begin{itemize}
        \item \textbf{Verify Signature}: Check that $\SIG.\verify(P.\sigma_0, P.\sigma_1, P.\body) = \True$. This check is only performed if the transfer shape includes at least one \Sender{}.
        \item \textbf{Public Withdraw}: All the public addresses corresponding to the \Asset{s} in $P.\body.\sources$ have enough public balance (i.e. in the \PublicLedger{}) to withdraw the given \Asset{}.
        \item \textbf{Public Deposit}: All the public addresses corresponding to the \Asset{s} in $P.\body.\sinks$ exist.
        \item \textbf{Current Accumulated State}: The $\UTXOSet.\Output$ stored in each $P.\body.\senders$ is equal to current accumulated value, $\UTXOSet.\current(\Ledger.\utxos())$, for the current state of the \Ledger{}.
        \item \textbf{New \VoidNumber{s}}: All the \VoidNumber{s} in $P.\body.\senders$ are unique, and no \VoidNumber{} in $P.\body.\senders$ has already been stored in the $\Ledger.\VoidNumberSet$.
        \item \textbf{New \UTXO{s}}: All the \UTXO{s} in $P.\body.\receivers$ are unique, and no \UTXO{} in $P.\body.\receivers$ has already been stored on the ledger.
        \item \textbf{Verify \Transfer{}}: Check that the following relation holds:
            \begin{align*}
                \NIZK.\verify_\vk(& \\
                    &P.\sigma_0 \,||\, P.\body.\ID \,||\, P.\body.\sources \,||\, P.\body.\senders \,||\, P.\body.\receivers \,||\, P.\body.\sinks, \\
                    &P.\body.\pi \\
                ) = \True
            \end{align*}
            where $P.\sigma_0$ is included whenever the transfer shape includes at least one \Sender{} and $P.\body.\ID$ is included whenever the transfer shape includes at least one of \Source{s} or \Sink{s}.
    \end{itemize}
\end{definition}

\begin{definition}[\Ledger{} \Transfer{} Update]
    After checking that a given \TransferPost{} $P$ is valid, the \Ledger{} updates its state by performing the following changes:
    \begin{itemize}
        \item \textbf{Public Updates}: All the relevant public accounts on the \PublicLedger{} are updated to reflect their new balances using the \Source{s} and \Sink{s} present in $P$.
        \item \textbf{\UTXOSet{} Update}: The new \UTXO{s} are appended to the \UTXOSet{}.
        \item \textbf{\VoidNumberSet{} Update}: The new \VoidNumber{s} are appended to the \VoidNumberSet{}.
    \end{itemize}
\end{definition}

\lsubsection{Batched Transactions}{abstract-batched-transactions}

For \MantaPay{} participants to use the \Transfer{} protocol, they will need to keep track of the current state of their \zkAsset{s} and use them to build \TransferPost{s} to send to the \Ledger{}. The balance of any participant is the sum of the balances of their \zkAsset{s}, but this balance may be fragmented into arbitrarily many pieces, as each piece represents an independent asset that the participant received as the output of some \Transfer{}. To then spend a subset of their balance, the participant would need to accumulate all of the relevant fragments into a large enough \zkAsset{} to spend all at once, building a collection of \TransferPost{s} to send to the \Ledger{}.

\begin{algorithm*}
\caption{Batched Transaction Algorithm}
\begin{algorithmic}
    \Procedure {BuildBatch}{$\sk$, $\mathcal{B}$, $\total$, $\pk$}
        \State $B \gets \Sample(\total, \mathcal{B})$  \Comment{Samples coins from $\mathcal{B}$ that total at least $\total$}
        \If{$\len(B) = 0$}
            \State \textsf{return} $[\,]$ \Comment{Insufficient Balance}
        \EndIf
        \State $P \gets [\,]$ \Comment{Allocate a new list for \TransferPost{s}}
        \While{$\len(B) > N$} \Comment{While there are enough pairs to make another \Transfer}
            \State $A \gets [\,]$
            \For{$b \in \next(B, N)$} \Comment{Get the next $N$ pairs from $B$}
                \State $S \gets \BuildSenders_\sk(b)$
                \State $[acc, zs...] \gets \BuildAccumulatorAndZeroes_\sk(S)$ \Comment{Build a new accumulator and zeroes}
                \State $P \gets P + \TransferPost(\Transfer([], S, [acc, zs...], []))$
                \State $(A, Z) \gets (A + acc, Z + zs)$ \Comment{Save $acc$ for the next loop, $zs$ for the end}
            \EndFor
            \State $B \gets A + \remainder(B, N)$
        \EndWhile
        \State $S \gets \PrepareZeroes_\sk(N, B, Z, P)$ \Comment{Use $Z$ and \Mint{s} to make $B$ go up to $N$ in size.}
        \State $R \gets \BuildReceiver_\sk(\pk, S)$
        \State $[c, zs...] \gets \BuildAccumulatorAndZeroes_\sk(S)$
        \State \textsf{return} $P + \TransferPost(\Transfer([], S, [R, c, zs...], []))$
    \EndProcedure
\end{algorithmic}
\end{algorithm*}

Any wallet implementation should see that their users need not keep track of this complexity themselves. Instead, like a public ledger, the notion of a \emph{transaction} between one participant and another should be viewed as a single atomic action that the user can take, performing a withdrawl from their balance. To describe such a \emph{batched transaction}, we assume the existence of two transfer shapes\footnotemark{}: $\Mint$ with shape $(1, 0, 1, 0)$ and $\PrivateTransfer$ with shape $(0, N, N, 0)$ for some natural number $N > 1$.

\footnotetext{Other \Transfer{} accumulation algorithms are possible with different starting shapes.}

For a fixed spending key, $\sk : \SpendingKey$, and asset id, $\ID : \AssetId$, we are given a balance state, $\mathcal{B} : \FinSet\left(\Bool \times \F \times \AssetValue\right)$, a set of transparenct-blinder-balance triples for unspent assets, a total balance to withdraw, $\total : \AssetValue$, and a receiving key $\pk : \zkAddress$. We can then compute 
\[\textsc{BuildBatch}(\sk, \mathcal{B}, \total, \pk)\]
to receive a $\List(\TransferPost)$ to send to the ledger, representing the transfer of $\total$ to $\pk$.

If all of the \Transfer{s} are accepted by the ledger, the balance state $\mathcal{B}$ should be updated accordingly, removing all of the pairs which were used in the \Transfer{}. Wallets should also handle the more complex case when only some of the \Transfer{s} succeed in which case they need to be able to continue retrying the transaction until they are finally resolved. Since the only \Transfer{} which sends \zkAsset{s} out of the control of the user is the last one (and it recursively depends on the previous \Transfer{s}), then it is safe to continue from a partially resolved state with a simple retry of the $\textsc{BuildBatch}$ algorithm.

