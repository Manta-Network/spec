\lsection{Concrete Protocol}{concrete-protocol}

We define the instantiation of the abstract protocol in this section, but first some preliminary notes.
 
\subsection{Poseidon Permutation and Poseidon Hash}
 
The \Poseidon{} Permutation ($\Poseidon^{\pi}$) \cite{KRRS21USENIX} is a finite field cryptographic primitive that can be used to build many cryptographic primitives, like hash functions, commitment schemes, and symmetric encryption schemes. \Poseidon{} plays a fundamental role in simplifying the \Transfer{} protocol and reducing the overall cost of the Zero-Knowledge circuits. $\Poseidon^{\pi}$ is a family of permutation functions with the following type:
\[\Poseidon^{\pi}_k : \F \times \F^k \to \F^k\]
over some sufficiencly large finite field $\F$. The first distinguished field element is used as a domain separation element. For this purpose, we use the following hashing function to generate domain strings:
\[\HashToScalar(m) := \F.\truncate(\textsf{Blake2s}(m))\]
 
The \Poseidon{} hash function (without sponges) with the following type:
\[\Poseidon_k : \F \times \F^k \to \F\]

is defined as extracting the first finite field element out of $\Poseidon^{\pi}_k$.
 
We make use of \Poseidon{} for a few values of $k$ in the concrete protocol below.

\subsection{Elliptic Curve Cryptography}

Because our protocol relies on a cryptographic group which should be efficient in a Zero-Knowledge Proving System we choose an elliptic curve defined over the finite field $\F$ of the proving system. To use group elements in affine form we also define the projections:
\[\X : \G \to \F \,\,\,\text{and}\,\,\, \Y : \G \to \F\]
which we use below to insert group elements into field-only hash functions.
 
For this protocol, we use \texttt{BN254} as our outer (pairing-friendly) curve with scalar field $\F$ and \texttt{BabyJubJub}~\cite{eip2494} as our inner curve with scalar field $\mathbb{S}$. For this protocol, we call the inner curve $\mathbb{G}$.
 
\subsection{Concrete Cryptographic Schemes}

\begin{definition}[Commitment Schemes]
    The protocol features two different commitment schemes: $\COM^{\U}$ the \UTXO{} Commitment Scheme and $\COM^{\N}$ the Nullifier Commitment Scheme. Both commitment schemes use \Poseidon{} as the underlying cryptographic primitive. The \UTXO{} uses an arity-5 \Poseidon{} with the following mapping:
    \[\COM^{\U}(r, \pk, \asset) := \Poseidon_5(d, r, \X(\pk), \Y(\pk), \asset.\ID, \asset.\VALUE)\]
    where $d = \HashToScalar(\domainstr{manta-pay/1.0.0/com-utxo})$ the domain separation element.
    For the Nullifier Commitment Scheme we use an arity-3 \Poseidon{} with the following mapping:
    \[\COM^{\N}(\ak, h) := \Poseidon_3(d, \X(\ak), \Y(\ak), h)\]
    where $d = \HashToScalar(\domainstr{manta-pay/1.0.0/com-vn})$ the domain separation element.
\end{definition}

\begin{definition}[Hash Functions]
    The protocol features two additional hash functions: $\HH^{\vk}$ the viewing key derivation function and $H^{\U}$ the \UTXO{} hash function. Both hash functions use \Poseidon{} as the underlying cryptographic primitive. The viewing key derivation function uses an arity-2 \Poseidon{}
    \[\HH^{\vk}(\ak) := \Poseidon_2(d, \X(\ak), \Y(\ak))\]
    where $d = \HashToScalar(\domainstr{manta-pay/1.0.0/vk-hash})$ the domain separation element. Since the target type of this hash function is the scalar field $\mathbb{S}$ instead of $\F$, we reduce the result modulo the order of $\mathbb{S}$.
    For the \UTXO{} hash function we use an arity-4 \Poseidon{}
    \[H^{\U}(t, \pa, \cm) := \Poseidon_4(d, t, \pa.\ID, \pa.\VALUE, \cm)\]
    where $d = \HashToScalar(\domainstr{manta-pay/1.0.0/utxo-hash})$ the domain separation element.
\end{definition}

\begin{definition}[Signature Scheme]
    For the signature scheme we use Schnorr signature over the inner curve, $\mathbb{G}$.
\end{definition}

\begin{definition}[Authenticated Encryption Scheme]
    For $\AUTHin$ and $\AUTHout$ we use the $\Poseidon$ permutation as the permutation of a duplex sponge \cite{duplexsponge} encryption protocol.
\end{definition}

\begin{definition}[Dynamic Cryptographic Accumulator]
    For $\DCA$, we use a Merkle Tree with $\Poseidon_2$ as the inner node combining hash function and no leaf hash function. It is safe to omit the leaf hash function in this case because the leaf values are already the outputs of a hash function and cannot be directly controlled.
\end{definition}

\begin{definition}[Non-Interactive Zero-Knowledge Proving System] \label{def:concrete-zkp}
    For $\NIZK$, the protocol can use any non-interactive zero-knowledge proving system like Groth16 \cite{KRRS21USENIX} and/or PLONK/PLONKUP~\cite{plonk,plonkup}. 
\end{definition}

\subsection{\AssetValue{} Bounds Check}

In order to implement the balanced transfer relation one needs to ensure that the amount of input value is equal to the amount of output value. However, since we're working over finite fields, the na\"ive arithmetic wraps past zero and is vulnerable to range-based attacks. Instead we constrain every \AssetValue{} to be less than some bound $\mathcal{V}$ and that every sum over those values is also less than $\mathcal{V}$. Since we're using \texttt{BN254} we are safe to use $\mathcal{V} = 2^{128}$.
