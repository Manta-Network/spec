\documentclass[a4paper]{article}

\usepackage[utf8]{inputenc}
\usepackage[T1]{fontenc}
\usepackage[margin=0.8in]{geometry}
\usepackage[
    colorlinks=true,
    linkcolor=blue,
    filecolor=blue,
    urlcolor=blue,
    pdfpagemode=FullScreen,
]{hyperref}
\usepackage{algorithm}
\usepackage{algpseudocode}
\usepackage{amsfonts}
\usepackage{amsmath}
\usepackage{amssymb}
\usepackage{amsthm}
\usepackage{commath}
\usepackage{float}
\usepackage{parskip}
\usepackage{tikz-cd}
\usepackage{wrapfig}

\newcommand{\subsubsubsection}[1]{\paragraph{#1}}
\newcommand{\subsubsubsubsection}[1]{\subparagraph{#1}}
\setcounter{secnumdepth}{4}
\setcounter{tocdepth}{4}

\def\sectionautorefname{$\S$}
\def\subsectionautorefname{$\S$}
\def\subsubsectionautorefname{$\S$}
\def\subsubsubsectionautorefname{$\S$}

\newcommand{\lsection}[2]{\def\sectionlabel{#2} \section{#1}\label{sec:#2}}
\newcommand{\lsubsection}[2]{\def\sectionlabel{#2} \subsection{#1}\label{sec:#2}}
\newcommand{\lsubsubsection}[2]{\def\sectionlabel{#2} \subsubsection{#1}\label{sec:#2}}
\newcommand{\lsubsubsubsection}[2]{\def\sectionlabel{#2} \subsubsubsection{#1}\label{sec:#2}}
\newcommand{\lsubsubsubsubsection}[2]{\def\sectionlabel{#2} \subsubsubsubsection{#1}\label{sec:#2}}
\newcommand{\TODO}[1]{{\color{red}\textbf{TODO}}: #1}

\floatstyle{boxed} 
\restylefloat{figure}

\newtheorem{definition}{Definition}[section]
\newtheorem*{definition*}{Definition}

\newcommand{\Alice}{{\textsf{Alice}}}
\newcommand{\AssetId}{{\textsf{AssetId}}}
\newcommand{\AssetValue}{{\textsf{AssetValue}}}
\newcommand{\Asset}{{\textsf{Asset}}}
\newcommand{\Bob}{{\textsf{Bob}}}
\newcommand{\COM}{{\textsf{COM}}}
\newcommand{\CRH}{{\textsf{CRH}}}
\newcommand{\Charlie}{{\textsf{Charlie}}}
\newcommand{\CheckProof}{{\textsf{CheckProof}}}
\newcommand{\Ciphertext}{{\textsf{Ciphertext}}}
\newcommand{\Concat}{{\textsf{Concat}}}
\newcommand{\Decrypt}{{\textsf{Decrypt}}}
\newcommand{\DerivePublic}{{\textsf{DerivePublic}}}
\newcommand{\Derive}{{\textsf{Derive}}}
\newcommand{\Encrypt}{{\textsf{Encrypt}}}
\newcommand{\EntropySource}{{\textsf{EntropySource}}}
\newcommand{\Enumerate}{{\textsf{enumerate}}}
\newcommand{\FAIR}{{\textsf{FAIR}}}
\newcommand{\GetAccumulatorAndZeroes}{{\textsf{GetAccumulatorAndZeroes}}}
\newcommand{\GetProof}{{\textsf{GetProof}}}
\newcommand{\HASH}{{\textsf{HASH}}}
\newcommand{\Input}{{\textsf{Input}}}
\newcommand{\Key}{{\textsf{Key}}}
\newcommand{\MantaDAP}{{\Manta{}_{\texttt{DAP}}}}
\newcommand{\MantaPay}{{\textsf{MantaPay}}}
\newcommand{\Manta}{{\textsc{Manta}}}
\newcommand{\Mint}{{\textsf{Mint}}}
\newcommand{\Next}{{\textsf{Next}}}
\newcommand{\Option}{{\textsf{Option}}}
\newcommand{\Output}{{\textsf{Output}}}
\newcommand{\PRF}{{\textsf{PRF}}}
\newcommand{\Plaintext}{{\textsf{Plaintext}}}
\newcommand{\PrepareZeroes}{{\textsf{PrepareZeroes}}}
\newcommand{\PrivateTransfer}{{\textsf{PrivateTransfer}}}
\newcommand{\Prop}{{\textsf{Prop}}}
\newcommand{\Prove}{{\textsf{Prove}}}
\newcommand{\PublicAddress}{{\textsf{PublicAddress}}}
\newcommand{\PublicLedger}{{\textsf{PublicLedger}}}
\newcommand{\Receiver}{{\textsf{Receiver}}}
\newcommand{\Reclaim}{{\textsf{Reclaim}}}
\newcommand{\Sample}{{\textsf{Sample}}}
\newcommand{\SeededRng}{{\textsf{SeededRng}}}
\newcommand{\Sender}{{\textsf{Sender}}}
\newcommand{\ShieldedAddress}{{\textsf{ShieldedAddress}}}
\newcommand{\ShieldedAssetPool}{{\textsf{ShieldedAssetPool}}}
\newcommand{\ShieldedIdentity}{{\textsf{ShieldedIdentity}}}
\newcommand{\Sink}{{\textsf{Sink}}}
\newcommand{\Some}{{\textsf{Some}}}
\newcommand{\Source}{{\textsf{Source}}}
\newcommand{\Symm}{{\textsf{Symm}}}
\newcommand{\TransferPost}{{\textsf{TransferPost}}}
\newcommand{\Transfer}{{\textsf{Transfer}}}
\newcommand{\TrapdoorDistribution}{{\textsf{TrapdoorDistribution}}}
\newcommand{\Trapdoor}{{\textsf{Trapdoor}}}
\newcommand{\Type}{{\textsf{Type}}}
\newcommand{\UTXO}{{\textsf{UTXO}}}
\newcommand{\Value}{{\textsf{Value}}}
\newcommand{\Verify}{{\textsf{Verify}}}
\newcommand{\VoidNumber}{{\textsf{VoidNumber}}}
\newcommand{\Void}{{\textsf{Void}}}
\newcommand{\allocated}{{\textsf{allocated}}}
\newcommand{\commit}{{\textsf{commit}}}
\newcommand{\decrypt}{{\textsf{decrypt}}}
\newcommand{\encrypt}{{\textsf{encrypt}}}
\newcommand{\hash}{{\textsf{hash}}}
\newcommand{\len}{{\textsf{len}}}
\newcommand{\note}{{\textsf{note}}}
\newcommand{\public}{{\textsf{public}}}
\newcommand{\pub}{{\textsf{pub}}}
\newcommand{\remainder}{{\textsf{remainder}}}
\newcommand{\secret}{{\textsf{secret}}}
\newcommand{\spend}{{\textsf{spend}}}
\newcommand{\spent}{{\textsf{spent}}}

\title{\textbf{\MantaPay{} Protocol Specification}\\ v0.4.0}
\author{Shumo Chu and Brandon H. Gomes}
\date{\today}

\begin{document}
    
\maketitle

\begin{abstract}
    \MantaPay{} is an implementation of a \emph{decentralized anonymous payment} scheme based on the $\MantaDAP$ protocol outlined in the original \href{https://eprint.iacr.org/2021/743.pdf}{\Manta{} whitepaper}.
\end{abstract}
    
\tableofcontents

\lsection{Introduction}{introduction}

\lsection{Notation}{notation}

\lsection{Concepts}{concepts}

\subsection{Assets}

The \Asset{} is the fundamental currency object in the \MantaPay{} protocol. An asset $a : \Asset$ is a tuple
\[a = (a.\textsf{id}, a.\textsf{value}) : \AssetId \times \AssetValue\]
The \MantaPay{} protocol is a \emph{decentralized anonymous payment} scheme which facilitiates the private ownership and private transfer of \Asset{} objects. The \AssetId{} field encodes the type of currency being used, and the \AssetValue{} encodes how many units of that currency are being used, in the standard base unit of that currency.

Whenever an \Asset{} is being used in a public setting, we simply refer to it as an \Asset{}, but when the \AssetId{} and/or \AssetValue{} of a particular \Asset{} is meant to be hidden from public view, we refer to the \Asset{} as either, \emph{secret}, \emph{private}, \emph{hidden}, or \emph{shielded}.

\Asset{s} form the basic units of \emph{transactions} which consume \Asset{s} on input, transform them, and return \Asset{s} on output. To preserve the economic value stored in \Asset{s}, the sum of the input \AssetValue{s} must balance the sum of the output \AssetValue{s}, and all assets in a single transaction must have the same \AssetId{}\footnotemark{}. 

\footnotetext{It is beyond the scope of this paper to discuss transactions with inputs and outputs that feature different \AssetId{s}, like those that would be featured in a \emph{decentralized anonymous exchange}.}

\lsubsection{Addresses}{addresses}

In order for participants in the \MantaPay{} protocol to send and receive \Asset{s}, they must create secret and public \emph{addresses} according to an \emph{address scheme}. For \MantaPay{}, the address scheme consists of a \emph{spending key} $sk$, a \emph{viewing key} $vk$, and a \emph{public key} $pk$. The keys have the following uses/properties:

\begin{itemize}
    \item Access to a public key $pk$ represents the ability to send \Asset{s} to the owner of the associated $sk$.
    \item Access to a viewing key $vk$ represents the ability to reveal shielded \Asset{} information for \Asset{s} belonging to the owner of the associated $sk$.
    \item Access to a spending key $sk$ represents the ability to spend \Asset{s} that were received under the associated public key $pk$.
\end{itemize}

See \autoref{sec:abstract-encryption} and \autoref{sec:addresses-and-key-components} for more information on how these keys are constructed and used for spending, viewing, and receiving.

\subsection{Ledger}

\begingroup
\setlength{\columnsep}{20pt}

\begin{wrapfigure}{r}[-10pt]{0.45\textwidth}
    \begin{center}
    \begin{tikzcd}
        \ShieldedAssetPool & & \\
            & \Transfer
                \arrow[r, "\spend"]
                \arrow[lu, leftrightarrow, start anchor = north west, end anchor = south east]
                \arrow[ld, leftrightarrow, start anchor = south west, end anchor = north east]
            & \Void \\
        \PublicLedger & &
    \end{tikzcd}
    \end{center}
    \caption{Lifecycle of an \Asset{}.}
\end{wrapfigure}

Ensuring that \Asset{s} maintain their economic value is not only dependent on transactions preserving inputs and outputs, but also that \Asset{s} are not \emph{double-spent}. The \emph{double-spending problem} can be solved by using a public ledger\footnotemark{} that keeps track of the flow of \Asset{s} from one participant to the other. Unfortunately, using a public ledger alone does not allow participants to remain anonymous, so \MantaPay{} extends the public ledger by adding a special account called the \ShieldedAssetPool{}. The \ShieldedAssetPool{} is responsible for keeping track of the \Asset{s} which have been anonymized by the protocol.

\Asset{s} can be in one of three states, \public{} (tracked by the \PublicLedger{}), \allocated{} (spendable subset of the \ShieldedAssetPool{}), or \spent{} (voided \Asset{s}). By way of the \autoref{sec:transfer-protocol} \Transfer{} Protocol, \Asset{s} can be sent to and from the \PublicLedger{} and the \ShieldedAssetPool{}.

The \ShieldedAssetPool{} is made up of four parts:

\begin{enumerate}
    \item \ShieldedAssetPool{} Balance: The \MantaPay{} ledger contains a collection of \Asset{s} which represent the combined economic value of the \ShieldedAssetPool{} and the \PublicLedger{}. The \ShieldedAssetPool{} Balance is the subset of this total value that has been anonymized by the \MantaPay{} protocol.
    \item \autoref{sec:ledger-utxo-set} \UTXO{} Set: A collection of claims to subsets of the \ShieldedAssetPool{}, each owned by participants of the \MantaPay{} protocol.
    \item \autoref{sec:ledger-encrypted-notes} Encrypted Notes: For each \UTXO{} there is a matching encrypted \note{} which contains information necessary to spend the \Asset{}, which is commited in the \UTXO{}, but can only be decrypted by the recipient of the \Asset{}, specifically, the correct viewing key $vk$. See \autoref{sec:addresses} for more.
    \item \autoref{sec:ledger-void-number-set} \VoidNumber{} Set: A collection of commitments keeping track of those \UTXO{s} which have participated in exactly one instance of the \Transfer{} Protocol.
\end{enumerate}

An \Asset{} is in the \public{} state if it belongs to the \PublicLedger{}. An \Asset{} is in the \allocated{} state if a \UTXO{} for the \Asset{} is a member of the \UTXO{} Set, but its matching \VoidNumber{} is \textbf{not} in the \VoidNumber{} Set. An \Asset{} is in the \spent{} state if it was \allocated{} in the past, but its matching \VoidNumber{} is now in the \VoidNumber{} Set.

The operation of the different parts of the \ShieldedAssetPool{} is elaborated in the following subsections.

\endgroup

\footnotetext{A public (or private) ledger is not enough to solve the \emph{double-spending problem}. A \emph{consensus mechanism} is also required to ensure that all participants agree on the current state of the ledger. The \emph{consensus mechanism} that secures the \MantaPay{} ledger is beyond the scope of this paper.}

\lsubsubsection{\UTXO{} Set}{ledger-utxo-set}

\lsubsubsection{Encrypted Notes}{ledger-encrypted-notes}

\lsubsubsection{\VoidNumber{} Set}{ledger-void-number-set}

\lsection{Abstract Protocol}{abstract-protocol}

\subsection{Abstract Cryptographic Schemes}

\subsubsection{Commitments}

A \emph{commitment scheme} $\COM$ is defined by the following schema:
\begin{align*}
    \Trapdoor             &: \Type \\
    \Input                &: \Type \\
    \Output               &: \Type \\
    \TrapdoorDistribution &: \mathcal{D}(\Trapdoor) \\
    \commit               &: \Trapdoor \times \Input \to \Output
\end{align*}
with the properties:

\begin{itemize}
    \item Binding: It is infeasible to find an $x, y : \Input$ and $r, s: \Trapdoor$ such that $x \ne y$ and $\commit(r, x) = \commit(s, y)$.
    \item Hiding: For all $x, y : \Input$, the distributions $\set{\commit(r, x)\,|\,r \sim \TrapdoorDistribution}$ and \\ $\set{\commit(r, y)\,|\,r \sim \TrapdoorDistribution}$ are \emph{computationally indistinguishable}.
\end{itemize}

\textbf{Notation}: For convenience we refer to $\COM.\commit(r, x)$ by $\COM_r(x)$.

\subsubsection{Hash Functions}

A \emph{hash function} $\CRH$ is defined by the following schema:
\begin{align*}
    \Input  &: \Type \\
    \Output &: \Type \\
    \hash   &: \Input \to \Output
\end{align*}
with the properties:

\begin{itemize}
    \item Pre-Image Resistance: For a given $y : \Output$, it is infeasible to find $x : \Input$ such that $\hash(x) = y$.
    \item Collision Resistance: It is infeasible to find an $x_1, x_2 : \Input$ such that $x_1 \ne x_2$ and $\hash(x_1) = \hash(x_2)$.
\end{itemize}

\textbf{Notation}: For convenience we refer to $\CRH.\hash(x)$ by $\CRH(x)$.

\lsubsubsection{Encryption}{abstract-encryption}

A \emph{symmetric encryption echeme} $\Symm$ is defined by the following schema:
\begin{align*}
    \Key &: \Type \\
    \Plaintext &: \Type \\
    \Ciphertext &: \Type \\
    \encrypt &: \Key \times \Plaintext \to \Ciphertext \\
    \decrypt &: \Key \times \Ciphertext \to \Option(\Plaintext)
\end{align*}
with the properties:

\begin{itemize}
    \item Valid Decryption: $\decrypt(k, \encrypt(k, p)) = \Some(p)$
    \item \TODO{hiding? one-time encryption security?}
\end{itemize}

\subsubsection{Zero-Knowledge Proving Systems}

\lsubsection{Addresses and Key Components}{addresses-and-key-components}

\lsubsection{\Transfer{} Protocol}{transfer-protocol}

\subsubsection{\Sender{s}}

\subsubsection{\Receiver{s}}

\subsubsection{\Transfer{s}}

\subsubsection{\TransferPost{s}}

\lsection{Concrete Protocol}{concrete-protocol}

\subsection{Conventions}

\subsection{Constants}

\subsection{Concrete Cryptographic Schemes}

\subsubsection{Commitments}

\subsubsection{Hash Functions}

\subsubsection{Encryption}

\subsubsection{Zero-Knowledge Proving Systems}

\subsubsubsection{Groth16}

\subsubsubsection{PLONK}

\lsection{Differences from $\MantaDAP$}{differences}

\subsection{Reusable Addresses}

\subsection{\Transfer{} Circuit Unification}

\lsection{Acknowledgements}{acknowledgements}

\lsection{References}{references}

\end{document}

