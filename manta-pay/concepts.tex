\lsection{Concepts}{concepts}

\lsubsection{\zkAsset{s}}{concept-zkasset}

The \zkAsset{} is the fundamental currency object in the \MantaPay{} protocol. An asset $a : \zkAsset$ is a tuple
\[a = (a.\ID, a.\VALUE) : \AssetId \times \AssetValue\]
where the \AssetId{} encodes the type of currency stored in $a$ and the \AssetValue{} encodes how many units of that currency are stored in $a$.  \MantaPay{} is a \emph{decentralized anonymous payment} protocol which facilitiates the private ownership and private transfer of \zkAsset{s}. 

\zkAsset{s} are the basic building-blocks of \emph{transactions} which consume a set of input \zkAsset{s} and produce a set of transformed output \zkAsset{s}. To preserve the economic value stored in \zkAsset{s}, the sum of the input \AssetValue{s} must balance the sum of the output \AssetValue{s}, and all assets in a single transaction must have the same \AssetId{}\footnotemark{}.\footnotetext{It is beyond the scope of this paper to discuss transactions with inputs and outputs that feature different \AssetId{s}, like those that would be featured in a \emph{decentralized anonymous exchange}.} This is called a \emph{balanced transfer}: no value is created or destroyed in the process. The \MantaPay{} protocol uses a distributed algorithm called \Transfer{} to perform balanced transfers and ensure that they are valid.

\lsubsection{\UTXO{s}}{concept-utxo}

But \zkAsset{s} are not private on their own. A \UTXO{} is a container for a \zkAsset{} that hides its value and its owner and is the main object that \MantaPay{} uses to transfer the spending power of \zkAsset{s} between different protocol participants. A \UTXO{} is a cryptographic commitment along with some associated data that represents a spendable subset of an account stored in the protocol. In the \MantaPay{} protocol, \UTXO{s} come in two flavors, \emph{opaque} and \emph{transparent}. The \emph{opaque} \UTXO{s} are completely private and they do not reveal the owner or underlying asset contained in them, whereas \emph{transparent} \UTXO{s} reveal the underlying asset but not the owner. The \emph{opaque} \UTXO{} is used for the private transfer of \zkAsset{s} and the \emph{transparent} \UTXO{} is used to give programability to \zkAsset{s} whenever the \MantaPay{} protocol lives in the same environment as other smart contracts by allowing contracts to control the \AssetId{} and \AssetValue{} stored in the \emph{transparent} \UTXO{}.

\lsubsection{\Nullifier{s}}{concept-nullifier}

One of the important ways that privacy is preserved for \zkAsset{s} across many transactions is that the exact transaction where a \UTXO{} is spent is not known to the public. Instead, only the owner of the \zkAsset{}, or anyone with the appropriate viewing key, can know this information. The \Nullifier{} is another cryptographic commitment that takes the place of the \UTXO{} when it is spent and it is cryptographically hard for any particular \UTXO{} to be derived from its \Nullifier{}.

\lsubsection{\zkAddress{es}}{concept-zkaddress}

In order for \MantaPay{} participants to receive \zkAsset{s} via the \Transfer{} protocol, they create \emph{zk-addresses} which they use as identifiers to represent them on the ledger.

\begin{center}
    \vspace{1em}
    \begin{mdframed}[leftmargin=0.2\textwidth, rightmargin=0.2\textwidth]
        \begin{center}
            \begin{tikzcd}
                & \sk \arrow[ld] \arrow[rd] & & & \\
                \sk_\alpha \arrow[rd] & & \ak \arrow[ld] \arrow[r] & \vk \arrow[r] & \pk \\
                & \ak_\alpha & & &
            \end{tikzcd}
        \end{center}
    \end{mdframed}
    \vspace{-1em}
    \captionof{figure}{Key Schedule for \MantaPay{}.}
\end{center}

\MantaPay{} uses four kinds of keys all derived from a base secret, spending key $\sk$, which give the following kinds of privileged access in the protocol:

\begin{itemize}
    \item \textbf{\zkAddress{}} (\send{}): Access to the zk-address $\pk$ gives the user the right to send \zkAsset{s} to the owner of the associated $\sk$. 
    \item \textbf{Viewing Key} (\view{}): Access to the viewing key $\vk$ gives the user the right to view all transactions for the owner of the associated $\sk$.
    \item \textbf{Proof Authorization Key} (\prove{}): Proof authorization key $\ak$ gives the user the right to build the $\Transfer$ proof on behalf of the owner of $\sk$. This key is used when delegating proof generation to a semi-trusted entity while still protecting the spending rights associated to the $\sk$, for example, if a hardware wallet holds $\sk$ it can ask a more capable computer to produce the $\Transfer$ proof for it without sending the spending rights off of the hardware wallet.
    \item \textbf{Spending Key} (\spend{}): Access to the spending key $\sk$ gives total control over the assets owned by this secret, including spending, proof generation, and viewing.
\end{itemize}

Participants in \MantaPay{} are represented by their zk-addresses, but they are not unique representations, since one participant may have access to more than one secret key. See \autoref{sec:addresses-and-key-components} for more information on how these keys are constructed and used for spending, proving, viewing, and receiving.

\lsubsection{\Note{s}}{concept-note}

The encrypted \Note{} is the primary means of communication in the \MantaPay{} protocol. For a \zkAddress{} owner to know that they have received a \zkAsset{} and can now spend it they decrypt \Note{s} with their viewing key to discover how much of an asset they have received and what information they need to spend it. The \Note{} is also used to keep track of the balances of an entire account over its transaction history.

There are two kinds of \Note{s} in the \MantaPay{} protocol, \emph{incoming} \Note{s} and \emph{outgoing} \Note{s}. The \IncomingNote{} is attached to every new \UTXO{} and contains the same \zkAsset{} as the \UTXO{} and also a secret randomizer used to hide the \UTXO{} commitment. The \OutgoingNote{} is attached to every new \Nullifier{} and contains the same \zkAsset{} as the \UTXO{} that the \Nullifier{} is marking. When performing accounting over a \zkAddress{} to measure how much of a particular \AssetId{} that address controls, the \AssetValue{} stored in the \IncomingNote{s} should be \emph{added} to the running total whereas the \AssetValue{} stored in the \OutgoingNote{s} should be \emph{subtracted} from the running total as they represent inflows and outflows respectively.

\lsubsection{\ShieldedPool{}}{concept-shielded-pool}

The \ShieldedPool{} is a data structure that contains the necessary data to enable the \MantaPay{} \Transfer{} protocol. The \ShieldedPool{} is made up of the following three general storage groups:

\begin{itemize}
    \item \UTXO{} Storage: Contains all of the \UTXO{s} that have ever been created along with their \IncomingNote{s}
    \item \Nullifier{} Storage: Contains all of the \Nullifier{s} that have ever been created along with their \OutgoingNote{s}
    \item Public Pool Account: The public account of the pool itself that holds a backing of all the \zkAsset{s} held in the \UTXO{s} in the pool. Depositing into or withdrawing out of the pool has to go through this account.
\end{itemize}

There are two general requirements on the \UTXO{} and \Nullifier{} storage items:

\begin{enumerate}
    \item Fast non-membership query for \UTXO{s} and \Nullifier{s}
    \item Fast insertion and insertion-order iteration over $(\UTXO{}, \IncomingNote{})$ and $(\Nullifier{}, \OutgoingNote{})$ pairs
\end{enumerate}

In order to satisfy both of these requirements we have the following breakdown of the storage:

\begin{itemize}
    \item \UTXO{} Storage:
        \begin{itemize}
            \item $\UTXOSet: \UTXO \to \Bool$
            \item $\UTXOStorageInsertionOrder: \mathbb{N} \to (\UTXO, \IncomingNote)$
        \end{itemize}
    \item \Nullifier{} Storage:
        \begin{itemize}
            \item $\NullifierSet: \Nullifier \to \Bool$
            \item $\NullifierStorageInsertionOrder: \mathbb{N} \to (\Nullifier, \OutgoingNote)$
        \end{itemize}
\end{itemize}

where we use the sets for fast non-membership checks and the insertion order maps for insertion-order preserving insertion and iteration.

%\lsubsection{Ledger}{concept-ledger}
%
%Preserving the economic value of \zkAsset{s} requires more than just balanced transfers. It also requires that \zkAsset{s} are owned by exactly one address at a time, namely, that the ability to spend a \zkAsset{} can be proved before a transfer and revoked after a transfer. It is not simply the \emph{information-content} of an \zkAsset{} that should be transfered, but the \emph{ability to spend the asset in the future}, which should be transfered. To enforce this second invariant we can use a public ledger\footnotemark{} that keeps track of the movement of \zkAsset{s} from one participant to another.\footnotetext{A public (or private) ledger is not enough to solve the \emph{provable-ownership problem} or the \emph{double-spending problem}. A \emph{consensus mechanism} is also required to ensure that all participants agree on the current state and state transformations of the ledger. The design and specification of the consensus mechanism that secures a \MantaPay{} ledger is beyond the scope of this paper.} Unfortunately, using a public ledger alone does not allow participants to remain anonymous, so \MantaPay{} extends the public ledger by adding a special account called the \emph{shielded asset pool} which is responsible for keeping track of the \Asset{s} which have been anonymized by the protocol. We denote the three ledger types in the protocol as follows: the public ledger as \PublicLedger{}, the shielded asset pool as \ShieldedAssetPool{}, and the combined ledger we denote \Ledger{}.
%
%The \ShieldedAssetPool{} is made up of three parts that are used to enforce the balanced transfer of \Asset{s} among anonymous participants:

%\begin{enumerate}
%    \item \autoref{sec:ledger-utxo-set} \UTXOSet{}: The \UTXOSet{} is a collection of ownership claims to subsets of the \ShieldedAssetPool{} (called \UTXO{s}), each one refering to an allocated \Asset{} transfered to a participant of the protocol.
%    \item \autoref{sec:ledger-encrypted-notes} \EncryptedNote{s}: For every \UTXO{} there is a matching \EncryptedNote{} which contains information necessary to spend the \Asset{}, which can be used to \emph{provably reconstruct} the \UTXO{} convincing the \Ledger{} of unique ownership. The \EncryptedNote{} can only be decrypted by the recipient of the \Asset{} or the designated viewer of the \UTXO{}, specifically, the correct viewing key $\vk$. See \autoref{sec:addresses} for more.
%    \item \autoref{sec:ledger-nullifier-set} \NullifierSet{}: The \NullifierSet{} is a collection of commitments, like \UTXO{s}, but which track the \emph{spent state} of an \Asset{} and are used to prove to the \Ledger{} that an \Asset{} is spent \emph{exactly one time}. 
%\end{enumerate}

% The operation of these different parts of the \ShieldedAssetPool{} is elaborated in the following subsections.

% \lsubsubsection{\UTXO{s} and the \UTXOSet{}}{ledger-utxo-set}

%An \emph{unspent transaction output}, or \UTXO{} for short, represents a claim to the output of a balanced transfer which has otherwise \emph{not yet been spent}. Every balanced transfer can produce some number of \emph{public outputs}, represented by \Asset{s}, and/or \emph{private outputs}, represented by \UTXO{s}, and these \UTXO{s} are stored in the \UTXOSet{} of the \ShieldedAssetPool{}. A \UTXO{} can only be claimed by the participant who owns the underlying \Asset{}, where ownership means \emph{knowledge of the correct spending key} and the \Transfer{} protocol requires that all inputs to a balanced transfer \emph{prove} that they own a \UTXO{} which the \ShieldedAssetPool{} has already seen in the past. The \UTXOSet{} is \emph{append-only} since it represents the past state of \emph{unspent} \Asset{s}. \UTXO{s} can only be added to the \UTXOSet{} as outputs in the execution of a \Transfer{} which the \Ledger{} checks for correctness.

%\lsubsubsection{\EncryptedNote{s}}{ledger-encrypted-notes}
%
%In order to find out what \Asset{} a \UTXO{} is connected to, every \UTXO{} comes with an associated \EncryptedNote{} which stores two pieces of information, the underlying \Asset{}, and an ephemeral public key, a value which allows the new owner of the \Asset{} to reconstruct the \UTXO{}. Being able to \emph{provably reconstruct} a correct \UTXO{} is a prerequisite to ownership and the ability to spend the \Asset{} in the future. Once a participant spends an \Asset{} that they can decrypt, they build a new \EncryptedNote{} for the next participant that they sent their \Asset{s} to, so that they can then spend it, and so on. This is called the \emph{in-band secret distribution}.

% \lsubsubsection{\Nullifier{s} and the \NullifierSet{}}{ledger-nullifier-set}

%Once the ability to spend an \Asset{} is extracted from a $(\UTXO, \EncryptedNote)$ pair, the \ShieldedAssetPool{} requires another commitment in order to spend the \Asset{}, transfering it to another participant. This commitment, called the \Nullifier{}, represents the revocation of the right to spend the \Asset{} in the future, and ensures that the same \Asset{} cannot be spent twice. Like the \UTXOSet{}, the \NullifierSet{} is \emph{append-only} since it represents the past state of \emph{spent} \Asset{s}. \Nullifier{s} can only be added to the \NullifierSet{} as inputs in the execution of a \Transfer{} which the \Ledger{} checks for correctness.

