\lsection{Concepts}{concepts}
\subsection{Assets}

The \Asset{} is the fundamental currency object in the \MantaPay{} protocol. An asset $a : \Asset$ is a tuple
\[a = (a.\ID, a.\VALUE) : \AssetId \times \AssetValue\]
where the \AssetId{} encodes the type of currency stored in $a$ and the \AssetValue{} encodes how many units of that currency are stored in $a$.  \MantaPay{} is a \emph{decentralized anonymous payment} protocol which facilitiates the private ownership and private transfer of \Asset{} objects. 

Whenever an \Asset{} is being used in a public setting, we simply refer to it as an \Asset{}, but when the \AssetId{} and/or \AssetValue{} of a particular \Asset{} is meant to be hidden from public view, we refer to the \Asset{} as either, \emph{secret}, \emph{private}, \emph{hidden}, or \emph{shielded}.

\Asset{s} are the basic building-blocks of \emph{transactions} which consume a set of input \Asset{s} and produce a set of transformed output \Asset{s}. To preserve the economic value stored in \Asset{s}, the sum of the input \AssetValue{s} must balance the sum of the output \AssetValue{s}, and all assets in a single transaction must have the same \AssetId{}\footnotemark{}.\footnotetext{It is beyond the scope of this paper to discuss transactions with inputs and outputs that feature different \AssetId{s}, like those that would be featured in a \emph{decentralized anonymous exchange}.} This is called a \emph{balanced transfer}: no \AssetValue{} is created or destroyed in the process. The \MantaPay{} protocol uses a distributed algorithm called \Transfer{} to perform balanced transfers and ensure that they are valid.

\lsubsection{Addresses}{addresses}

In order for \MantaPay{} participants to receive \Asset{s} via the \Transfer{} protocol, they create a \emph{shielded addresses} which they use as identifiers to represent them on the ledger.

\begin{center}
    \vspace{1em}
    \begin{mdframed}[leftmargin=0.2\textwidth, rightmargin=0.2\textwidth]
        \begin{center}
            \begin{tikzcd}
                & \sk \arrow[ld] \arrow[rd] & & & \\
                \sk_\alpha \arrow[rd] & & \ak \arrow[ld] \arrow[r] & \vk \arrow[r] & (\textsf{D}, \pk_\textsf{D}) \\
                & \ak_\alpha & & &
            \end{tikzcd}
        \end{center}
    \end{mdframed}
    \vspace{-1em}
    \captionof{figure}{Key Schedule for \MantaPay{}.}
\end{center}

\MantaPay{} uses four kinds of keys all derived from a base secret key $\sk$, which give the following kinds of privileged access in the protocol:

\begin{itemize}
    \item \textbf{Shielded Address} (\send{}): Access to the shielded address $(\textsf{D}, \pk_\textsf{D})$ gives the user the right to send \Asset{s} to the owner of the associated $\sk$. The diversifier $\textsf{D}$ allows the owner of a given $\sk$ key to generate many shielded addresses with the same backing spend authority.
    \item \textbf{Viewing Key} (\view{}): Access to the viewing key $\vk$ gives the user the right to view all transactions for the owner of the associated $\sk$.
    \item \textbf{Proof Authorization Key} (\prove{}): Access to proof authorization key $\ak$ gives the user the right to build the transaction proof on behalf of the owner of $\sk$. In order to delegate proof generation to another party, the owner of the secret key generates a randomizer $\alpha$ and sends it to the prover which generates the proof. The owner then signs the transaction against $\ak_\alpha$ with their randomized key $\sk_\alpha$ which proves that they have knowledge of $\sk$.
    \item \textbf{Spending Key} (\spend{}): Access to the spending key $\sk$ gives total control over the assets owned by this secret, including spending, proof generation, and viewing.
\end{itemize}

Participants in \MantaPay{} are represented by their addresses, but they are not unique representations, since one participant may have access to more than one secret key. See \autoref{sec:addresses-and-key-components} for more information on how these keys are constructed and used for spending, proving, viewing, and receiving.

\subsection{Ledger}

Preserving the economic value of \Asset{s} requires more than just balanced transfers. It also requires that \Asset{s} are owned by exactly one address at a time, namely, that the ability to spend an \Asset{} can be proved before a transfer and revoked after a transfer. It is not simply the \emph{information-content} of an \Asset{} that should be transfered, but the \emph{ability to spend the asset in the future}, which should be transfered. To enforce this second invariant we can use a public ledger\footnotemark{} that keeps track of the movement of \Asset{s} from one participant to another.\footnotetext{A public (or private) ledger is not enough to solve the \emph{provable-ownership problem} or the \emph{double-spending problem}. A \emph{consensus mechanism} is also required to ensure that all participants agree on the current state and state transformations of the ledger. The design and specification of the consensus mechanism that secures the \MantaPay{} ledger is beyond the scope of this paper.} Unfortunately, using a public ledger alone does not allow participants to remain anonymous, so \MantaPay{} extends the public ledger by adding a special account called the \emph{shielded asset pool} which is responsible for keeping track of the \Asset{s} which have been anonymized by the protocol. We denote the three ledger types in the protocol as follows: the public ledger as \PublicLedger{}, the shielded asset pool as \ShieldedAssetPool{}, and the combined ledger we denote \Ledger{}.

\begin{center}
    \vspace{1em}
    \begin{mdframed}[leftmargin=0.2\textwidth, rightmargin=0.2\textwidth]
        \begin{center}
            \begin{tikzcd}
                && \UTXOSet \arrow[ld, "\spend" description, bend right] \\
                \PublicLedger \arrow[r, "\mint"', bend right]
                & \textbf{\Transfer} \arrow[l, "\reclaim"', bend right]
                \arrow[rd, "\spend" description]
                \arrow[ru, "\allocate" description, bend right] & \\
                && \VoidNumberSet
            \end{tikzcd}
        \end{center}
    \end{mdframed}
    \vspace{-1em}
    \captionof{figure}{Life cycle of an \Asset{}.}
\end{center}

The \ShieldedAssetPool{} is made up of three parts that are used to enforce the balanced transfer of \Asset{s} among anonymous participants:

\begin{enumerate}
    \item \autoref{sec:ledger-utxo-set} \UTXOSet{}: The \UTXOSet{} is a collection of ownership claims to subsets of the \ShieldedAssetPool{} (called \UTXO{s}), each one refering to an allocated \Asset{} transfered to a participant of the protocol.
    \item \autoref{sec:ledger-encrypted-notes} \EncryptedNote{s}: For every \UTXO{} there is a matching \EncryptedNote{} which contains information necessary to spend the \Asset{}, which can be used to \emph{provably reconstruct} the \UTXO{} convincing the \Ledger{} of unique ownership. The \EncryptedNote{} can only be decrypted by the recipient of the \Asset{} or the designated viewer of the \UTXO{}, specifically, the correct viewing key $\vk$. See \autoref{sec:addresses} for more.
    \item \autoref{sec:ledger-void-number-set} \VoidNumberSet{}: The \VoidNumberSet{} is a collection of commitments, like \UTXO{s}, but which track the \emph{spent state} of an \Asset{} and are used to prove to the \Ledger{} that an \Asset{} is spent \emph{exactly one time}. 
\end{enumerate}

The operation of these different parts of the \ShieldedAssetPool{} is elaborated in the following subsections.

\lsubsubsection{\UTXO{s} and the \UTXOSet{}}{ledger-utxo-set}

An \emph{unspent transaction output}, or \UTXO{} for short, represents a claim to the output of a balanced transfer which has otherwise \emph{not yet been spent}. Every balanced transfer can produce some number of \emph{public outputs}, represented by \Asset{s}, and/or \emph{private outputs}, represented by \UTXO{s}, and these \UTXO{s} are stored in the \UTXOSet{} of the \ShieldedAssetPool{}. A \UTXO{} can only be claimed by the participant who owns the underlying \Asset{}, where ownership means \emph{knowledge of the correct spending key} and the \Transfer{} protocol requires that all inputs to a balanced transfer \emph{prove} that they own a \UTXO{} which the \ShieldedAssetPool{} has already seen in the past. The \UTXOSet{} is \emph{append-only} since it represents the past state of \emph{unspent} \Asset{s}. \UTXO{s} can only be added to the \UTXOSet{} as outputs in the execution of a \Transfer{} which the \Ledger{} checks for correctness.

\lsubsubsection{\EncryptedNote{s}}{ledger-encrypted-notes}

In order to find out what \Asset{} a \UTXO{} is connected to, every \UTXO{} comes with an associated \EncryptedNote{} which stores two pieces of information, the underlying \Asset{}, and an ephemeral public key, a value which allows the new owner of the \Asset{} to reconstruct the \UTXO{}. Being able to \emph{provably reconstruct} a correct \UTXO{} is a prerequisite to ownership and the ability to spend the \Asset{} in the future. Once a participant spends an \Asset{} that they can decrypt, they build a new \EncryptedNote{} for the next participant that they sent their \Asset{s} to, so that they can then spend it, and so on. This is called the \emph{in-band secret distribution}.

\lsubsubsection{\VoidNumber{s} and the \VoidNumberSet{}}{ledger-void-number-set}

Once the ability to spend an \Asset{} is extracted from a $(\UTXO, \EncryptedNote)$ pair, the \ShieldedAssetPool{} requires another commitment in order to spend the \Asset{}, transfering it to another participant. This commitment, called the \VoidNumber{}, represents the revocation of the right to spend the \Asset{} in the future, and ensures that the same \Asset{} cannot be spent twice. Like the \UTXOSet{}, the \VoidNumberSet{} is \emph{append-only} since it represents the past state of \emph{spent} \Asset{s}. \VoidNumber{s} can only be added to the \VoidNumberSet{} as inputs in the execution of a \Transfer{} which the \Ledger{} checks for correctness.
